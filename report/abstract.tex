\chapter*{Abstract}\label{sec:abs}\thispagestyle{fancy}
\addcontentsline{toc}{chapter}{Abstract}
%%
High-confinement mode (H-mode) is a promising reference scenario for ITER. But we are still facing major issues because of instabilities. They expel periodically some of the energy, which can damage the device. These instabilities are called the edge localized modes (ELM) and are not yet fully theoretically understood.

The present work is a study on the profiles evolution in between ELMs and on the ELM effects. This may help to have a better understanding of the conditions before the ELM. We use the simulations as theoretical tool.

For the purpose of the simulations, we build an H-mode $\chi_e$ profile according to a standard L-mode one that we truncate at the edge to create a transport barrier. This gives a good agreement with the experimental data.

Several scaling laws were successfully used. The first one is the energy confinement time scaling which was used for the thermal diffusivity to scale the temperature profile. A scaling between the core and pedestal energies was found recently. It was used to compute the pedestal $\chi_e$ to scale the temperature pedestal, which was successful. Finally, we used a scaling for transport barriers which links the density gradient length to that of the temperature to compute the density in the pedestal. It was already found to be good in TCV electron internal transport barriers and in ASDEX Upgrade H-mode pedestals.

Looking at the MHD stability parameters, it was found that for our reference case, ELMs are not likely to be triggered by the time evolution of the pressure gradient and the current density profiles in our model, as these are only varying significantly during the first millisecond after the crash, and are almost constant during the long remaining time until the next crash.

Studying different cases, we investigate the behavior of the plasma when replacing the edge heating by central one to observe the influence of the heating profile, but no significant difference was found, neither in the MHD stability parameters.

Further we change the particle diffusion coefficient to compare the dynamic behavior of the density. Slowing down the density dynamic behavior also slows down the pressure one, this can be seen on the MHD stability parameters. We also vary the ELM period to compare to the change due to the variation of the particle diffusivity. It was found that there may be a sort of relation between the particle diffusivity and the ELM period at least for the density, since both cases change the density recovery time with respect to the ELM period.

A last case considered is doubling the radial ELM interaction range. This is done in order to observe the difference to the reference simulation that takes the density top of pedestal as ELM range, and to compare the spatial range influenced by the MHD activity and the one by the transport improvement. It was found that the MHD stability parameters in the pedestal exhibit a different behavior with the pressure gradient starting to increase very fast.
