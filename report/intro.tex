\chapter{Introduction}\thispagestyle{fancy}
%%
Fusion may be our future source of energy. For now, we are not able to achieve it in an economically viable way because we do not understand fully what happens in our tokamaks. A long time ago, Lawson was also working on fusion research. He predicted that we will face some problems he did not know at that moment. This was very smart of him. He also predicted from where those problems will arise: ``Conduction loss is difficult to treat in a general way, since it depends on the geometry of the system, its density and temperature distribution, and also the wall material.'' \cite{lawson}\ Actually the wall material has not much to do with the losses, or indirectly. Nowadays we are facing problems coming from within the plasma, instabilities and turbulence.

As the research went further on fusion, we discovered some limitations to our reactors. Looking for a way to overcome them, it was once discovered in ASDEX a regime where the confinement is improved by about a factor two \cite{iterNews}. This was done twenty-nine years ago. This operation mode was not predicted, but it appeared for a sufficient input power. It is called the H-mode, standing for \emph{High-confinement mode}, in opposition to the \emph{Low-confinement mode} (L-mode). Since then this promising operational mode has been studied extensively.

We then discovered that one of the features of H-mode is the instabilities called edge localized modes (ELMs). ELMs expel periodically some particles and energy. They are somehow useful, because it helps the plasma expel the impurities, and not to increase too much the density. But there are many types of them, and some expel very large quantities of energy. The expelled particles and energy onto the vessel. This may cause destructive erosion of the plasma-facing components that could make the device not viable.

In order to have a better knowledge of the physics in H-mode, here is studied the electronic transport in the inter-ELM phase. This can be done through theoretical studies to avoid the machine deterioration. We will do a brief overview of the limits responsible for the ELMs, but this work focuses on the profiles evolution in between ELMs and on the ELM effects. This may help us to better understand the conditions before an ELM.

The next step in fusion research is being built in France, the famous International Thermonuclear Experiment Reactor (ITER). Its operational scenario is an ELMy H-mode. It is therefore of prime importance to have a better knowledge of the transport in the inter-ELM phase.

This work will first recall some theory basis that are needed. We will do an overview of the magnetohydrodynamics theory and instabilities. Then we will speak about the plasma confinement, more specifically of the transport phenomena, leading to the H-mode description. This study being theoretical, we then explain the tool used in this work and how the implementation has been achieved. Finally, different cases were studied and comparisons between these results are presented. We will study the inter-ELM profiles, time traces and the MHD stability diagrams.
